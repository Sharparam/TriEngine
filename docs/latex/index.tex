2\-D general-\/purpose engine in C\#/\-Open\-G\-L

\subsection*{I\-R\-C}

\hyperlink{namespace_tri_devs}{Tri\-Devs} has an I\-R\-C channel, feel free to hop in if you have a question about anything\-: {\bfseries Server\-:} irc.\-kottnet.\-net {\bfseries Port\-:} 6667, 6697 (S\-S\-L) {\bfseries Channel\-:} \hyperlink{namespace_tri_devs}{Tri\-Devs}

The channel topic contains further info.

\subsection*{License}

Copyright © 2013 by \href{https://github.com/Sharparam}{\tt Adam Hellberg}, \href{https://github.com/Vijfhoek}{\tt Sijmen Schoon} and \href{https://github.com/anidude}{\tt Preston Shumway}.

Tri\-Engine2\-D is licensed under the \href{http://opensource.org/licenses/MIT}{\tt M\-I\-T License}, more info can be found in the {\bfseries L\-I\-C\-E\-N\-S\-E} file.

\subsection*{Contributing}

You are free to fork this project and make your own changes, as long as you follow the M\-I\-T License.

If you want to make a pull request, please do so to the \href{https://github.com/TriDevs/TriEngine2D}{\tt main project} and not any of the \char`\"{}official\char`\"{} forks.

For your pull request to be accepted, please follow our coding style\-:
\begin{DoxyItemize}
\item Indent with 4 spaces, not tabs.
\item Curly braces placed on next line.
\item All {\bfseries public} methods, accessors and members must be properly documented.
\item Use sensible variable names that describe what they are for.
\item Method declarations written as\-:
\end{DoxyItemize}

```c\# public void Hello(string world) ```


\begin{DoxyItemize}
\item If your method accepts many parameters, it can be useful to put parameters on separate lines, as per this style\-:
\end{DoxyItemize}

```c\# public void Hello(string world, bool print) ```


\begin{DoxyItemize}
\item Please write tests for your code (not strictly required, but it's a plus)
\end{DoxyItemize}

By looking through the current source code, you should be able to get a good understanding of the formatting we use.

If you're using Visual Studio, you can change the indent behaviour by going to\-: {\bfseries Tools} -\/$>$ {\bfseries Options} -\/$>$ {\bfseries Text Editor} -\/$>$ {\bfseries C\#} -\/$>$ {\bfseries Tabs} and make sure \char`\"{}\-Insert spaces\char`\"{} is checked.

If you write tests for your code, please place these tests in their own project\-: \char`\"{}\&lt; $<$strong$>$\-Namespace$<$/strong$>$ \&gt;.\-Tests\char`\"{}, create said project if it does not exist (of type Class Library).

We use N\-Unit as test framework, feel free to use something else if you want to, but make sure you document what framework you are using and that it is freely available for anyone to obtain.

\subsection*{Dependencies}

Tri\-Engine2\-D depends on \href{http://logging.apache.org/log4net/}{\tt log4net}, which is included in the {\bfseries libs/log4net} folder.

Tri\-Engine2\-D depends on \href{http://www.opentk.com/}{\tt Open\-T\-K}, this is not included and you will have to build/install it yourself. Open\-T\-K depends on Open\-G\-L drivers being installed, they are usually in your normal video card drivers.

Tri\-Engine2\-D depends on \href{http://json.codeplex.com/}{\tt Json.\-N\-E\-T}, this is not included, but is specified in the Nu\-Get package config. If you \href{http://docs.nuget.org/docs/workflows/using-nuget-without-committing-packages#Using_NuGet_without_committing_packages_to_source_control}{\tt properly configure your Nu\-Get settings}, Nu\-Get will automatically download Json.\-N\-E\-T when building any projects that depend on it.

If you want to run the tests you will need to have \href{http://www.nunit.org/}{\tt N\-Unit} installed. 